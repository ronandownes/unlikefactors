\documentclass{beamer}

\usetheme{Madrid}
\usecolortheme{default}

\title{Multiplying Unlike-Pattern Binomials}
\subtitle{Binomial × Binomial With Dissimilar Patterns}
\author{Mr Downes}
\date{}

\begin{document}

\frame{\titlepage}

% ============================================================
\begin{frame}{When Two Binomials Do NOT Share a Pattern}
\LARGE
\begin{itemize}
    \item A pattern is the structural form of a bracket.
    \item Example of similar pattern: \((x+3)\) and \((x-5)\).
    \item Example of dissimilar pattern: \((x + 4)\) and \((y + 2)\).
    \item When patterns differ, no like terms will appear.
\end{itemize}
\end{frame}

% ============================================================
\begin{frame}{Our Starting Example}
\LARGE
Multiply:
\[
(x + 3)(y + 2)
\]
\begin{itemize}
    \item One bracket is in \(x\) only.
    \item The other bracket is in \(y\) only.
    \item Unlike factor patterns.
    \item Therefore: no grouping; use \textbf{partial products}.
\end{itemize}
\end{frame}

\begin{frame}{The Partial Product Method}
\LARGE
Multiply each term in the first bracket by each term in the second bracket.
\[
(x + 3)(y + 2)
\]

\[
\begin{aligned}
\textbf{F:}\;& x \cdot y &= xy \\[4pt]
\textbf{O:}\;& x \cdot 2 &= 2x \\[4pt]
\textbf{I:}\;& 3 \cdot y &= 3y \\[4pt]
\textbf{L:}\;& 3 \cdot 2 &= 6
\end{aligned}
\]

\[
\Rightarrow\ xy + 2x + 3y + 6
\]
\end{frame}

% ============================================================
\begin{frame}{Key Observation}
\LARGE
\[
(x + 3)(y + 2) = xy + 2x + 3y + 6
\]

\begin{itemize}
    \item No common factor patterns
    \item No like terms.
    \item No grouping.
    \item Unlike (pattern) factors imply ( \(\implies\) ) all 4 unlike terms.
\end{itemize}
\end{frame}

% ============================================================
\begin{frame}{General Form}
\LARGE
For any dissimilar binomials
\[
(x + a)(y + b)
\]
where \(x\) appears in one bracket and \(y\) in the other:
\[
(x + a)(y + b)
= xy + bx + ay + ab
\]

This is the clean Singapore–style structure.
\end{frame}

% ============================================================
\begin{frame}{Class Rule}
\LARGE
\[
\textbf{If the brackets do not share a pattern, do not group.}
\]

\[
\textbf{Multiply each term by each term.}
\]

\bigskip
This is the safest and most general method for binomial × binomial.
\end{frame}

% ============================================================
\begin{frame}{Practice Tasks}
\LARGE
Expand the following (no grouping expected):

\[
(x+5)(y+1)
\]

\[
(2x-3)(y+4)
\]

\[
(p+7)(q+2)
\]
\end{frame}

\end{document}
